\documentclass[12pt,a4paper]{article}

% Magyar nyelv, kódolás, betűtípus (Times-szerű)
\usepackage[magyar]{babel}
\usepackage[utf8]{inputenc}
\usepackage[T1]{fontenc}
\usepackage{newtxtext,newtxmath} % Times New Roman-hoz közeli betűtípus

% Oldalbeállítás, sorkizárt, 1,5 sortáv
\usepackage{geometry}
\geometry{margin=3cm}
\usepackage{setspace}
\onehalfspacing
\usepackage{ragged2e}
\justifying

% Címsorok formázása
\usepackage{titlesec}
% Címsor1: 14pt, balra igazított, félkövér (1)
\titleformat{\section}
  {\bfseries\large}   % 14pt körüli méret 12pt alapnál
  {\thesection.}       % "1." jelölés
  {0.5em}{}
% Címsor2: 12pt, balra igazított, félkövér (1.1)
\titleformat{\subsection}
  {\bfseries\normalsize}
  {\thesubsection.}
  {0.5em}{}

% Hiperhivatkozott tartalomjegyzék
\usepackage[hidelinks]{hyperref}

% Képek
\usepackage{graphicx}
\usepackage{float}

% Kódlisták
\usepackage{xcolor}
\usepackage{listings}

\lstset{
  basicstyle=\ttfamily\small,   % más betűtípus a kódhoz
  numbers=left,
  numberstyle=\tiny,
  keywordstyle=\color{blue},
  commentstyle=\color{gray},
  stringstyle=\color{red},
  breaklines=true,
  frame=single
}

\begin{document}

%========================================
% Címlap
%========================================
\begin{titlepage}
    \centering
    \vspace*{3cm}
    {\Huge\bfseries Jegyzőkönyv \\[0.5cm]}
    {\Large Webes adatkezelő környezetek \\[0.3cm]}
    {\Large Féléves feladat \\[0.3cm]}
    {\Large Gyógyszertár adatkezelő rendszere\par}

    \vfill
    \begin{flushright}
        \textbf{Készítette:} Mizerák Bence\\[0.3cm]
        \textbf{Neptunkód:} MO0VEY\\[0.3cm]
        \textbf{Dátum:} 2025. november
    \end{flushright}

    \vspace{2cm}
    {\large Miskolc, 2025}
\end{titlepage}

%========================================
% Tartalomjegyzék (hiper-hivatkozott)
%========================================
\tableofcontents
\newpage

%========================================
\section{Bevezetés}
%========================================
Ez a jegyzőkönyv a \emph{Webes adatkezelő környezetek} tárgy féléves feladatát
mutatja be. A projekt célja egy gyógyszertár adatkezelő rendszerének
modellje volt XML, XML Schema (XSD) és Java DOM API felhasználásával.

Az XML a strukturált adatok platformfüggetlen cseréjére alkalmas
szabványos formátum. A séma (XSD) biztosítja, hogy az XML dokumentum
szerkezete, kulcsai és idegen kulcsai jól definiáltak legyenek, míg a
Java DOM API lehetővé teszi az adatok programozott beolvasását,
módosítását és kiírását.

%========================================
\section{A rendszer áttekintése}
%========================================

\subsection{Domain és fájlstruktúra}
A modell egy gyógyszertár rendszerének főbb adatait tartalmazza. A
központi XML állomány a \texttt{MO0VEY\_XML.xml}, amelynek gyökéreleme a
\texttt{<GyogyszertarRendszer>}. A főbb részegységek:

\begin{itemize}
  \item \texttt{Gyogyszerek/Gyogyszer}: gyógyszerek törzse
        (név, hatóanyag, receptköteles, kategória, leírás, mellékhatások),
  \item \texttt{Kiszerelesek/Kiszereles}: egyes gyógyszerek kiszerelései
        (forma, mennyiség, ár, egység),
  \item \texttt{Betegsegek/Betegseg}: betegségek leírása,
        figyelmeztetések, tünetek,
  \item \texttt{BetegsegGyogyszerKapcsolatok/BetegsegGyogyszer}:
        mely gyógyszer mely betegségre ajánlott,
  \item \texttt{Rendelesek}: vevői rendelések fej- és tételadatai,
  \item \texttt{Szallitok/Szallito}: beszállítók adatai,
  \item \texttt{Beszerzesek}: beszerzési számlák fej- és tételadatai.
\end{itemize}

A fenti XML-hez tartozó sémafájl a \texttt{MO0VEY\_XMLSchema.xsd}, míg a
DOM alapú Java programok a \texttt{MO0VEYDOMParse} mappában találhatók:
\texttt{MO0VEYDomRead.java}, \texttt{MO0VEYDomWrite.java},
\texttt{MO0VEYDomModify.java}, \texttt{MO0VEYDomQuery.java}. A módosított
állapotot a \texttt{MO0VEY\_XML\_modified.xml} tartalmazza.

%========================================
\section{ER modell}
%========================================
A logikai adatmodellt először egy entitás--kapcsolat (ER) diagramon
ábrázoltuk. A fő entitások: \emph{Gyogyszer}, \emph{Kiszereles},
\emph{Betegseg}, \emph{Szallito}, \emph{Rendeles\_Fej},
\emph{Rendeles\_Tetel}, \emph{Beszerzes\_Fej}, \emph{Beszerzes\_Tetel},
valamint a \emph{BetegsegGyogyszer} kapcsoló entitás.

A kapcsolatok között megjelennek az 1:N és N:M típusok is. Például egy
\emph{Gyogyszer}-hez több \emph{Kiszereles} tartozhat (1:N), míg egy
\emph{Betegseg}-hez több \emph{Gyogyszer} is ajánlható, és fordítva (N:M),
amit a \emph{BetegsegGyogyszer} entitás közvetít.

Az ER modellt az \ref{fig:er-modell} ábra szemlélteti.

\begin{figure}[H]
  \centering
  % Itt az ER modell képfájlja (például ER_Feher.png)
  \includegraphics[width=0.9\textwidth]{ER_Feher.png}
  \caption{1. ábra: A gyógyszertár ER modellje}
  \label{fig:er-modell}
\end{figure}

%========================================
\section{XDM modell}
%========================================
Az ER diagramból XDM (XML Data Model) modellt készítettünk. Ebben minden
entitás egy XML elemnek felel meg, a kulcs tulajdonságok pedig
attribútumként jelennek meg. A gyökér a \texttt{GyogyszertarRendszer}
elem, ez tartalmazza a fenti részfákat (\texttt{Gyogyszerek},
\texttt{Kiszerelesek}, \texttt{Betegsegek}, stb.).

Külön figyelmet kapott, hogy minden \texttt{*\_ID} azonosító
\emph{attribútumként}, ne pedig gyerekelemként szerepeljen, így a
kulcs/idegen kulcs kapcsolatok természetesebben írhatók le az XSD-ben
\texttt{xs:key} és \texttt{xs:keyref} használatával.

Az XDM modell felépítését a \ref{fig:xdm-modell} ábra mutatja be.

\begin{figure}[H]
  \centering
  % Itt az XDM modell képfájlja (például XDM_Feher.png)
  \includegraphics[width=0.9\textwidth]{XDM_Feher.png}
  \caption{2. ábra: A gyógyszertár XDM modellje}
  \label{fig:xdm-modell}
\end{figure}

%========================================
\section{XML és XML Schema}
%========================================
Az XDM modell alapján készült el a \texttt{MO0VEY\_XML.xml} állomány. Az
azonosítók mindenhol attribútumként jelennek meg, például:

\begin{itemize}
  \item \texttt{<Gyogyszer Gyogyszer\_ID="G001"> ... </Gyogyszer>},
  \item \texttt{<Kiszereles Kiszereles\_ID="K001" Gyogyszer\_ID="G001"> ...},
  \item \texttt{<Betegseg Betegseg\_ID="B001"> ...},
  \item \texttt{<Rendeles\_Fej Rendeles\_ID="R001"> ...}.
\end{itemize}

Az XSD-ben ezekhez kulcsokat és idegen kulcsokat definiáltunk. Például
\texttt{GyogyszerKey} a gyógyszerek elsődleges kulcsa, míg a
\texttt{Kiszereles\_GyogyszerRef} biztosítja, hogy minden
\texttt{Kiszereles}@\texttt{Gyogyszer\_ID} egy létező
\texttt{Gyogyszer}@\texttt{Gyogyszer\_ID}-re hivatkozzon. Hasonló
kulcs--kulcsidegen kapcsolat van a rendelések és rendeléssorok, valamint a
beszerzések és beszerzéssorok között is.

Külön feladat volt, hogy minden ID-elem attribútummá alakítása után az
\texttt{xs:key} és \texttt{xs:keyref} kifejezésekben a mezőket
\texttt{@AttribútumNév} formában hivatkoztassuk, illetve a kulcsok
láthatósági tartományát úgy állítsuk be, hogy a vonatkozó keyref-ek ne
essenek ki a scope-ból.

%========================================
\section{Java DOM programok}
%========================================
A projekt második része a Java DOM API használata volt. Négy önálló
program készült:

\begin{itemize}
  \item \textbf{MO0VEYDomRead}: beolvassa az XML-t és fastruktúraként
        kiírja a konzolra,
  \item \textbf{MO0VEYDomWrite}: hasonló fa-jellegű kiírást végez, majd
        elmenti az eredményt a \texttt{MO0VEY1XML.xml} állományba,
  \item \textbf{MO0VEYDomModify}: módosítja az adatokat (új gyógyszer
        felvétele, megjegyzés változtatása, mennyiség és összeg
        újraszámolása, beszállító törlése) és elmenti a
        \texttt{MO0VEY\_XML\_modified.xml} fájlba,
  \item \textbf{MO0VEYDomQuery}: egy konkrét rendelés adatait gyűjti
        össze és számolja ki a végösszeget.
\end{itemize}

Azonosítókhoz a Java kódban már mindenhol attribútumként férünk hozzá,
például \texttt{getAttribute("Rendeles\_ID")} vagy
\texttt{getAttribute("Kiszereles\_ID")}. Az alábbi kódrészlet a
\texttt{MO0VEYDomQuery} osztályból mutatja be, hogyan történik egy
konkrét rendelés (\texttt{R001}) lekérdezése és összegzése. 

\begin{lstlisting}[language=java, caption={3. ábra: Részlet a MO0VEYDomQuery osztályból}]
private static void printRendelesById(Document doc, String rendelesId) {
    NodeList rendelFejek = doc.getElementsByTagName("Rendeles_Fej");
    for (int i = 0; i < rendelFejek.getLength(); i++) {
        Element fej = (Element) rendelFejek.item(i);
        String rId = fej.getAttribute("Rendeles_ID");
        if (rendelesId.equals(rId)) {
            String nev = fej.getElementsByTagName("Nev").item(0).getTextContent();
            String email = fej.getElementsByTagName("Email").item(0).getTextContent();
            String datum = fej.getElementsByTagName("Datum").item(0).getTextContent();
            String megjegyzes = fej.getElementsByTagName("Megjegyzes").item(0)
                                         .getTextContent();

            System.out.println("Rendeles ID: " + rId);
            System.out.println("Nev: " + nev);
            System.out.println("E-mail: " + email);
            System.out.println("Datum: " + datum);
            System.out.println("Megjegyzes: " + megjegyzes);

            int osszeg = 0;
            NodeList tetelek = doc.getElementsByTagName("Rendeles_Tetel");
            for (int j = 0; j < tetelek.getLength(); j++) {
                Element tetel = (Element) tetelek.item(j);
                String tetelRId = tetel.getAttribute("Rendeles_ID");
                if (rendelesId.equals(tetelRId)) {
                    int tetelOsszeg = Integer.parseInt(
                        tetel.getElementsByTagName("Osszeg").item(0).getTextContent()
                    );
                    osszeg += tetelOsszeg;
                }
            }
            System.out.println("Vegosszeg: " + osszeg + " Ft");
            break;
        }
    }
}
\end{lstlisting}

A kódrészlet jól szemlélteti, hogy a rendelés fej- és tételadatai
attribútumalapú azonosítók segítségével kapcsolódnak össze, ami
megfelel az XSD-ben definiált kulcs--kulcsidegen kapcsolatoknak.

%========================================
\section{Összegzés}
%========================================
A projekt során egy teljes gyógyszertári adatkezelő példát valósítottunk
meg az ER modelltől kezdve az XDM modellen, XML/XSD pároson át egészen a
Java DOM alapú feldolgozásig. A rendszer jól demonstrálja, hogyan lehet
az elméleti adatmodelleket gyakorlatban is használható, validálható XML
struktúrává, majd programozottan feldolgozható adathalmazzá alakítani.

\end{document}
